\documentclass[11pt]{article}
\usepackage{biblatex}
\bibliography{PMCS}

\begin{document}
\section*{Question 1} 

Introduction

The dynamics of a single-person academic project can be very different to those of a team project working in a commercial setting, where many more factors can affect the outcome.

In my personal project, there are no resources to be managed other than myself, very few cybersecurity risks and even less ethical issues.

However, by establishing the project scope, planning, and managing risks correctly, my project can benefit from project management techniques which are applicable to any project.

\begin{itemize}

    \item Project scope:
    Firstly, defining a project scope sets out clearly what the goals, timeframes and constraints of the project are. This technique will make it easier to apply other project management techniques, as well as allowing the project owner to have a clear view of what their goals are.
    \begin{itemize}
        \item Objective
        \item Time
        \item Constraints
    \end{itemize}

    \item Planning:
    Planning is another extremely useful technique for any project, as it allows the owner to see how the project is progressing, and adjust appropriately if it falls off schedule.
    By setting appropriate product-based milestones using product-led planning and keeping of the necessary steps with an activity list, one can have concrete indicators of how far the project is in its lifecycle and what the next step to take is.
    \begin{itemize}
        \item Product-led planning
        \item Activity list
        \item \textbf{Work flow diagram - not useful}
    \end{itemize}

    \item Risks: \textbf{Include examples of risks}
    Finally, every project has risks, and keeping track of them is the key to ensuring the project is not derailed by an unforeseen circumstance. Using risk assessment techniques, one can include uncertainty into their planning and actions to maximize the chances that the project is completed successfully.
    This is implemented in the risk cycle, where risks are continually assessed throughout the project and contingency plans are made to manage, mitigate or remediate the risks. By assigning an exposure value to each identified risk, the relative importance of each risk is put into perspective and will guide the management strategies that are identified.
    \begin{itemize}
        \item Risk cycle \-- continually reassess risks
        \item Risk exposure + management
    \end{itemize}


    \item Useless:
    Not all project management techniques apply to a single person project, however, and in my case my project does not require me to manage any resources, as the only resources involved in my project are me and my laptop.

    There are no major cybersecurity issues either, as I will be writing non safety-critical software with no network access and few permissions. Finally there are no ethics or legal issues for me to deal with
\begin{itemize}
    \item Stakeholders \textbf{maybe?}
    \item CSec
    \item Resource management
    \item Ethics
\end{itemize}
\end{itemize}
\end{document}
